\documentclass{article}
%%%%%%%%%%%%%%%%%%%%%%%  ПАКЕТЫ  %%%%%%%%%%%%%%%%%%%%%%%%%%%%%%%%%%%%%%%%%%%%%%
\usepackage{cmap}                               % Чтобы в PDF работал человеческий поиск
\usepackage[X2,T2A]{fontenc}                    % T2A = русская кодировка. X2 = яти
\usepackage[utf8]{inputenc}                     % Ввод в универсальной кодировке
\usepackage{setspace,soulutf8}      		        % Чтобы можно было менять межстрочный и межбуквенный интервалы
\usepackage{amsmath,amsfonts,amssymb,amsthm}    % Символы для математики
\usepackage{mathrsfs}                           % Символы для математики
\usepackage{dsfont}                             % Шрифт для знака индикатора
\usepackage[subfigure]{tocloft}  % Многоточие в оглавлении
\usepackage{array,multicol,multirow,bigstrut}   % Чтобы можно было делать в таблице колонки фиксированной ширины, слитные ячейки, вставлять strut'ы.
\usepackage{indentfirst}                        % Абзацный отступ везде
\usepackage[british,russian]{babel}                     % Русские переносы, тире, типографика, самодержавие, духовность!
\usepackage[perpage]{footmisc}                  % Сброс счётчика сносок на каждой странице
%\usepackage[pdftex,unicode,colorlinks=false,urlcolor=blue]{hyperref} % Ссылки в PDF
\usepackage{microtype}                          % Свешивающаяся пунктуация и подгонка белого пространства по правилу \pm 2 процента
\usepackage{textcomp}                           % Чтобы в формулах можно было русские буквы писать через \text{}
\usepackage[paper=a4paper,top=12.7mm, bottom=12.7mm,left=12.7mm,right=12.7mm,bindingoffset=6.6mm,includehead]{geometry} % Размеры листа и поля
\usepackage{xcolor}                             % Чтобы можно было цветные объекты вставлять
\usepackage[pdftex]{graphicx}                   % Чтобы вставились изображения
\usepackage{float,longtable}                    % Поддержка плавающих таблиц и рисунков
\usepackage[margin=0pt,font=small,labelfont=bf,labelsep=period]{caption} % Подписи таблиц и рисунком мелкие, жирные, с принятым в русской типографике разделителем.
\usepackage{rotating}                   % Создание своих акцентов, поворот объекта.
\usepackage{datetime}                           % Отображение времени
%\usepackage{embedfile}                          % Чтобы код LaTeXа включился как приложение в PDF-файл
\usepackage{xspace}
\usepackage{wrapfig,enumitem}                   % Обтекаемые текстом рисунки
\usepackage{mathtools}                          % В тексте используется smashoperator, чтобы избежать некрасивых пробелов вокруг сумм и пределов с большим подстрочником
\usepackage{cancel}                             % Красивое <<вычёркивание>> сокращающихся выражений
\usepackage{tikz,pgfplots}
\usepackage{subfigure}
\usepackage{fancyhdr}							%Колонтитулы

%%%%%%%%%%%%%%%%%%%%%%%%%%%%%%%%%%%%%%%%%%%%%%%%%%%%%%%%%%%%%%%%%%%%%%%%%%%%%%%



%%%%%%%%%%%%%%%%%%%%%%%  ПАРАМЕТРЫ  %%%%%%%%%%%%%%%%%%%%%%%%%%%%%%%%%%%%%%%%%%%
\setstretch{1}                          % Межстрочный интервал
\flushbottom                            % Эта команда заставляет LaTeX чуть растягивать строки, чтобы получить идеально прямоугольную страницу
\righthyphenmin=2                       % Разрешение переноса двух и более символов
%\pagestyle{plain}                       % Нумерация страниц снизу по центру.
\settimeformat{hhmmsstime}              % Формат времени с секундами
\widowpenalty=300                       % Небольшое наказание за вдовствующую строку (одна строка абзаца на этой странице, остальное --- на следующей)
\clubpenalty=3000                       % Приличное наказание за сиротствующую строку (омерзительно висящая одинокая строка в начале страницы)
\setlength{\parindent}{1.5em}           % Красная строка.
%\embedfile[desc={Исходный код этого файла для LaTeX2e}]{\jobname.tex}                % Включение кода в выходной файл
%\embedfile[desc={Обязательный стилевой файл}]{../hse.sty}
%\embedfile[desc={Обязательный стилевой файл}]{../hseafterpreamble.sty}
%\embedfile[desc={Андрей Викторович Костырка}]{../copyright.sty}
\setlength{\topsep}{0pt}                % Уничтожение верхнего отступа, если он где проявится
\renewcommand{\thesubsection}{\arabic{subsection}}
%%%%%%%%%%%%%%%%%%%%%%%%%%%%%%%%%%%%%%%%%%%%%%%%%%%%%%%%%%%%%%%%%%%%%%%%%%%%%%%



%%%%%%%% Это окружение, которое выравнивает по центру без отступа, как у простого center
\newenvironment{center*}{%
  \setlength\topsep{0pt}
  \setlength\parskip{0pt}
  \begin{center}
}{%
  \end{center}
}
%%%%%%%%%%%%%%%%%%%%%%%%%%%%%%%%%%%%%%%%%%%%%%%%%%%%%%%%%%%%%%%%%%%%%%%%%%%%%%%



%%%%%%%%%%%%%%%%%%%%% Правила переноса неизвестных системе слов (кто её знает?)

%%%%%%%%%%%%%%%%%%%%%%%%%%%%%%%%%%%%%%%%%%%%%%%%%%%%%%%%%%%%%%%%%%%%%%%%%%%%%%%



%%%%%%%%%%%%%%%%%%%%%%%%%%%%%%%%% Мои макрокоманды для облегчения набора текста
\newcommand{\E}{\mathbb{E}}                 % Матожидание
\newcommand{\e}{\varepsilon}                % эпсилон
\newcommand{\p}{\partial}                   % частная производная
\newcommand{\PP}{\mathbb{P}}                % Вероятность
\newcommand{\F}{\mathcal{F}}                % Алгебра F
\newcommand{\A}{\mathcal{A}}                % Алгебра A
\newcommand{\B}{\mathcal{B}}                % Алгебра B
\newcommand{\om}{\omega}                    % омега
\newcommand{\Om}{\Omega}                    % Омега
\newcommand{\stderr}{\ensuremath{\text{s.\hspace*{0.2ex}e.}}}
\newcommand{\bsbeta}{\boldsymbol{\beta}}
\newcommand{\calN}{\mathcal{N}}             % Курсивные буквы;
\newcommand{\calF}{\mathcal{F}}
\newcommand{\calI}{\mathcal{I}}
\newcommand{\calL}{\mathcal{L}}
\newcommand{\lagr}{\mathcal{L}}
\newcommand{\Lagr}{\mathcal{L}}
\newcommand{\fish}{\mathrm{F}}              % Символ распределения Фишера прямым шрифтом
\newcommand{\Fish}{\mathrm{F}}
\newcommand{\unif}{\mathrm{U}}
\newcommand{\Unif}{\mathrm{U}}
\newcommand{\taml}{\tilde a_{\text{ML}}}
\newcommand{\tbml}{\tilde b_{\text{ML}}}
\newcommand{\hypo}{\mathcal{H}}             % Символ гипотезы
\renewcommand{\phi}{\varphi}                % Чтоб фи писалась в соответствии с русской традицией
\newcommand{\ind}[1]{\mathds{1}_{\{#1\}}(\omega)} % Индикатор события [1] в множестве омега
\newcommand{\inds}[1]{\mathds{1}_{\{#1\}}}        % Индикатор события [1]
\renewcommand{\to}{\rightarrow}                   % Правильная стрелка вправо (<<стремится>>)
\newcommand{\sumin}{\sum\limits_{i=1}^n}          % Сумма от i=1 до n
\newcommand{\ofbr}[1]{\bigl( \{ #1 \} \bigr)}     % Большие круглые, нормальные фигурные скобки вокруг [1] (например, вероятность события)
\newcommand{\Ofbr}[1]{\Bigl( \bigl\{ #1 \bigr\} \Bigr)} % Больше больших круглые, большие фигурные скобки вокруг аргумента
\newcommand*{\circled}[1]{\tikz[baseline=(char.base)]{
            \node[shape=circle,draw,inner sep=1pt] (char) {#1};}}
\renewcommand{\le}{\leqslant}           % Правильное меньше или равно
\renewcommand{\leq}{\leqslant}           % Правильное меньше или равно
\renewcommand{\ge}{\geqslant}           % Правильное больше или равно
\renewcommand{\geq}{\geqslant}           % Правильное больше или равно
\newcommand{\br}[1]{\left( #1  \right)}    % Круглые скобки, подгоняемые по размеру аргумента
\newcommand{\fbr}[1]{\left\{ #1  \right\}} % Фигурные скобки, подгоняемые по размеру аргумента
\newcommand{\eqdef}{\mathrel{\stackrel{\text{def}}=}} % Знак <<равно по определению>>
\newcommand{\CONST}{\mathbb{C}}
\newcommand{\DF}{\ensuremath{\mathscr{D}\hspace{-0.3ex}\mathscr{F}}}
\newcommand{\ttilde}[1]{\tilde{\tilde{#1}}}
\newcommand{\kk}{\varkappa}
\newcommand{\assim}{\mathrel{\stackrel{\text{as}}\sim}}
\newcommand{\iid}{\text{i.\hspace{1pt}i.\hspace{1pt}d.}}
\renewcommand{\iff}{\,\Leftrightarrow\,}
\providecommand{\hence}{\Rightarrow}
\DeclareMathOperator{\const}{const}
\DeclareMathOperator{\Corr}{Corr}       % Оператор корреляции
\DeclareMathOperator{\corr}{Corr}
\DeclareMathOperator{\Cov}{Cov}         % Оператор ковариации
\DeclareMathOperator{\cov}{Cov}
\DeclareMathOperator{\Var}{Var}         % Оператор дисперсии
\DeclareMathOperator{\var}{Var}
\DeclareMathOperator{\rang}{rank}       % Оператор ранга
\DeclareMathOperator{\rank}{rank}
\DeclareMathOperator*{\plim}{plim}      % Оператор предела по вероятности
\DeclareMathOperator{\sign}{sgn}
\DeclareMathOperator{\sgn}{sgn}
\DeclareMathOperator{\diag}{diag}
\DeclareMathOperator{\Lin}{Lin}
\newcommand{\ENGs}[1]{\foreignlanguage{british}{#1}}
\newcommand{\ENG}{\selectlanguage{british}}
\newcommand{\RUS}{\selectlanguage{russian}}

\newcommand*{\tabvrulel}[1]{\multicolumn{1}{|c}{#1}} % Ячейка таблицы с центрированным содержимым и единичной прографкой слева
\newcommand*{\tabvruler}[1]{\multicolumn{1}{c|}{#1}} % Ячейка таблицы с центрированным содержимым и единичной прографкой справа
\newcommand{\fnnsp}{\hspace{-0.4em}} % Знак сноски принято ставить до всех знаков препинания, кроме ! ? ...
% А если есть возможность подвинуть низкий знак препинания под сноску, то для этого я и придумал \fnnsp


\newcommand{\II}{{\fontencoding{X2}\selectfont\CYRII}}   % I десятеричное (английская i неуместна)
\newcommand{\ii}{{\fontencoding{X2}\selectfont\cyrii}}   % i десятеричное
\newcommand{\EE}{{\fontencoding{X2}\selectfont\CYRYAT}}  % ЯТЬ
\newcommand{\ee}{{\fontencoding{X2}\selectfont\cyryat}}  % ять
\newcommand{\FF}{{\fontencoding{X2}\selectfont\CYROTLD}} % ФИТА
\newcommand{\ff}{{\fontencoding{X2}\selectfont\cyrotld}} % фита
\newcommand{\YY}{{\fontencoding{X2}\selectfont\CYRIZH}}  % ИЖИЦА
\newcommand{\yy}{{\fontencoding{X2}\selectfont\cyrizh}}  % ижица
%%%%%%%%%%%%%%%%%%%%%%%%%%%%%%%%%%%%%%%%%%%%%%%%%%%%%%%%%%%%%%%%%%%%%%%%%%%%%%%



%%%%%%%% Определение разрядки разреженного текста и задание красивых и притом регулируемых многоточий
% Зачем и почему описано в блоге http://kostyrka.ru/blog
\newdimen\ellipsiskern
\setlength{\ellipsiskern}{.1em}
\newdimen\ellipsiskernen
\setlength{\ellipsiskernen}{.2em}
\newcommand{\ldotst}{.\kern\ellipsiskern.\kern\ellipsiskern.}
\newcommand{\ldotse}{!\kern\ellipsiskern.\kern\ellipsiskern.}
\newcommand{\ldotsq}{?\kern\ellipsiskern\kern-.11em.\kern\ellipsiskern.}
\newcommand{\ldotsten}{.\kern\ellipsiskernen.\kern\ellipsiskernen.}
\newcommand{\ldotspen}{.\kern\ellipsiskernen.\kern\ellipsiskernen.\kern\ellipsiskernen\kern.15em.}
\newcommand{\ldotseen}{.\kern\ellipsiskernen.\kern\ellipsiskernen.\kern\ellipsiskernen\kern.15em!}
\newcommand{\ldotsqen}{.\kern\ellipsiskernen.\kern\ellipsiskernen.\kern\ellipsiskernen\kern.067em?}
%%%%%%%%%%%%%%%%%%%%%%%%%%%%%%%%%%%%%%%%%%%%%%%%%%%%%%%%%%%%%%%%%%%%%%%%%%%%%%%



%%%%%%%% Переопределение рубрикации в целях экономии места %%%%%%%%%%%%%%%%%%%%
%\makeatletter
%\renewcommand\subsection{\@startsection {subsection}{1}{\z@}%
%  {-2ex \@plus -1ex \@minus -.5ex}%
%  {.3ex \@plus.2ex \@minus -.1ex}%
%  {\normalfont\large\bfseries}}
%\makeatother
%
%\makeatletter
%\renewcommand\section{\@startsection {section}{1}{\z@}%
%  {-3.5ex \@plus -1ex \@minus -.2ex}%
%  {2.3ex \@plus.2ex}%
%  {\centering\normalfont\Large\bfseries\textsc}}
%\makeatother
%
%\makeatletter
%\renewcommand\part{\@startsection {part}{1}{\z@}%
%  {-3.5ex \@plus -1ex \@minus -.2ex}%
%  {2.3ex \@plus.2ex}%
%  {\centering\normalfont\LARGE\bfseries\textsc}}
%\makeatother
%%%%%%%%%%%%%%%%%%%%%%%%%%%%%%%%%%%%%%%%%%%%%%%%%%%%%%%%%%%%%%%%%%%%%%%%%%%%%%%



%%%%%%%%%%% Команда для переноса на следующую строку символов бинарных операций
\def\hm#1{#1\nobreak\discretionary{}{\hbox{$#1$}}{}}
%%%%%%%%%%%%%%%%%%%%%%%%%%%%%%%%%%%%%%%%%%%%%%%%%%%%%%%%%%%%%%%%%%%%%%%%%%%%%%%

\let\myfootnote\footnote
\renewcommand{\footnote}[1]{\myfootnote{\;#1}}

\let\oldleadsto\leadsto
\renewcommand{\leadsto}{\hspace*{1ex plus .3ex minus .2ex}\Rightarrow\hspace*{1ex plus .3ex minus .2ex}}

\input{mechmath.sty}
\newgeometry{paper=a4paper,top=12.7mm, bottom=12.7mm,left=12mm,right=12.7mm,bindingoffset=6.6mm,includefoot}
\begin{document}
\pagestyle{plain}
\section*{Контрольная на начало декабря третьего семестра у Алексея Константиновича}
\subsection*{Вариант 1}
\begin{enumerate}
\item Исследовать на равномерную сходимость
\begin{enumerate}
\item $\ds\int\limits_1^{+\infty}e^{-\alpha^2x}\frac{\cos x}{\sqrt x}\,dx$ на множестве $\alpha\in[-1,1]$.
\begin{enumerate}
\item Рассмотрим интеграл $\int\limits_1^{+\infty}\frac{\cos x}{\sqrt x}\,dx$.
\begin{itemize}
\item $\forall\ b>1\pau\Big|\int\limits_1^b\cos x\,dx\Big|=|\sin b - \sin 1|\le 2$.
\item $\frac1{\sqrt x}\searrow0$.
\end{itemize}
Значит, по признаку Дирихле $\int\limits_1^{+\infty}\frac{\cos x}{\sqrt x}\,dx$ сходится. Так как не зависит от параметра, сходится равномерно.
\item $e^{-\alpha^2 x}$ монотонна при любом $\alpha$ и $\forall\ x\in[1,+\infty),\forall\ \alpha\in[-1,1]\pau \big|e^{-\alpha^2 x}\big|\le 1$. 
\end{enumerate}
Значит, по признаку Абеля равномерной сходимости несобственного интеграла с параметром имеем равномерную сходимость.
\item $\ds I(\alpha)=\int\limits_0^{+\infty}e^{-\alpha^2(x^2+4)}\arctg \alpha\,dx$ на множестве $\alpha\in\R$. Посмотрим, а куда вообще сходится этот интеграл. Делаем замену $u=\alpha x$, считаем для $\alpha>0$.
\[I(\alpha)=e^{-4\alpha^2}\frac{\arctg \alpha}{\alpha}\intop_0^{\infty}e^{-u^2}\,du=\frac{\sqrt\pi\arctg \alpha}{2\alpha e^{4\alpha^2}}.\]
Заметим ещё $I(0)=0$. Тогда $\yo{\alpha}{0+}I(\alpha)=\yo{\alpha}{0+}\frac{\sqrt\pi\arctg \alpha}{2\alpha e^{4\alpha^2}}=\frac{\sqrt{\pi}}2\ne0=I(0)$. Значит, функция $I(\alpha)$ не является непрерывной. Следовательно, сходимость неравномерная.
\item $\ds\int\limits_0^{+\infty}\frac{\sin x^3}{1+x^{\alpha}}\,dx$ на множестве $\alpha>0$.
\end{enumerate}
\item Вычислить
\begin{enumerate}
\item $I(\alpha)=\ds\int\limits_0^{+\infty}\frac{\arctg\alpha x}{x(1+x^2)}\,dx=\int\limits_0^{1}\frac{\arctg\alpha x}{x(1+x^2)}\,dx+\int\limits_1^{+\infty}\frac{\arctg\alpha x}{x(1+x^2)}\,dx$.
\begin{itemize}
\item Интегрируем неотрицательную функцию, если $\alpha\ge0$. В нуле подынтегральная функция $f(x,\alpha)\sim \alpha$, в бесконечности "--- $f(x,\alpha)\sim \frac{\pi}{2x^3}$. Значит, интеграл сходится при любом фиксированном $\alpha\ge0$; видим, что $I(0)=0$. Отрицательные значения не будем рассматривать, так как функция нечётная.
\item $\forall\ b,d>0\pau f'_{\alpha}(x,\alpha)=\frac1{(1+\alpha^2x^2)(1+x^2)}\in C\big([0,b]\times[0,d]\big)$. Считаем интеграл для $|\alpha|\ne1$.
\begin{multline*}\intop_0^{+\infty}f'_{\alpha}(x,\alpha)\,dx=\frac1{1-\alpha^2}\bigg(\intop_0^{+\infty}\frac{dx}{1+x^2}-\intop_0^{+\infty}\frac{\alpha^2}{1+\alpha^2x^2}\,dx\bigg)=\\
=\frac1{1-\alpha^2}\bigg(\arctg x\Big|_0^{+\infty}-\alpha\arctg\alpha x\Big|_0^{+\infty}\bigg)=\frac1{1-\alpha^2}\left(\frac{\pi}2-\frac{\alpha\pi}2\right)=\frac{\pi}{2(1+\alpha)}.\end{multline*}
А если всё-таки $|\alpha|=1$, то вспоминаем такие обозначения $I_n=\ds\int\frac1{(1+x^2)^n}\,dx$.
\[I_1=\int\frac{1}{1+x^2}\,dx=\frac x{1+x^2}+2\int \frac{x^2}{(1+x^2)^2}\,dx=\frac x{1+x^2}+2I_1-2I_2;\quad\imp\quad I_2=\frac12I_1+\frac x{2(1+x^2)}.\]
Значит, нужный нам интеграл $\int\limits_0^{+\infty}f'_{\alpha}(x,1)\,dx=\frac12\arctg x\Big|_0^{+\infty}+\frac{x}{2(1+x^2)}\Big|_0^{+\infty}=\frac14\pi$. Производная получилась непрерывной. Посчитали интеграл $\int\limits_0^{+\infty}f'_{\alpha}(x,\alpha)\,dx$ с одной особенностью $+\infty$. Подынтегральное выражение мажорируется функцией $\frac1{1+x^4}$ при $x\ge1$. Значит, $\int\limits_1^{+\infty}f'_{\alpha}(x,\alpha)\,dx$ сходится равномерно по признаку Вейерштрасса на $\alpha\ge0$, а интеграл $\int\limits_0^1f'_{\alpha}(x,\alpha)\,dx$ собственный.
\end{itemize}
Значит, можно утверждать, что для $\alpha\ge0$ функция $I(\alpha)$ дифференцируема и $I'(\alpha)=\frac{\pi}{2(1+\alpha)}$. Тогда $I(\alpha)=\frac{\pi}2\ln(1+\alpha)+C$. $I(0)=0\imp C=0$. В ответ идёт $I(\alpha)=\sgn\alpha\frac{\pi}2\ln(1+|\alpha|)$.
\item $I=\ds\int\limits_0^1\frac{dx}{\sqrt[3]{1-x^3}}$. Замена $y=x^3$, $dy=3x^2$, $dx=\frac{dy}{3y^{\frac23}}$.
\[I=\frac13\intop_0^1y^{\frac13-1}(1-y)^{\frac23-1}\,dy=\frac13{\rm B}\left(1/3,2/3\right)=\frac{\pi}{3\sin\frac{\pi}3}=\frac{2\pi}{3\sqrt3}.\]
\end{enumerate}
\end{enumerate}\newpage
\subsection*{Вариант 2}
\begin{enumerate}
\item Исследовать на равномерную сходимость
\begin{enumerate}
\item $\ds\int\limits_1^{\infty}\frac{\cos\alpha x}{\sqrt x}\,dx$ на множестве $\alpha>0$.

Особенность у интеграла только одна.
\begin{itemize}
\item Частичные интегралы $\forall\ b\in(1,\infty)\pau\int\limits_1^b\cos\alpha x\,dx=\sin \alpha x\Big|_1^{b}=-\sin \alpha b$ равномерно ограничены ограничены по модулю единицей.
\item Для абсолютно произвольного $\alpha>0$ функция $\frac1{\sqrt x}$, вообще не зависящая от параметра, монотонно стремится к нулю при $x\to\infty$. Так как функция не зависит от параметра, $\frac1{\sqrt x}\rsH[(0,+\infty)]{x\to+\infty}0$.
\end{itemize}

Значит, по признаку Дирихле равномерной сходимости несобственного интеграла, зависящего от параметра, $\int\limits_1^{+\infty}\frac{\sin\alpha x}{\sqrt x}\,dx\rsH[(0,+\infty_)]{x\to+\infty}$.

\item $\ds\int\limits_0^{+\infty}\frac{x\arctg\frac{\alpha}x}{1+x^2}\,dx$ на множестве $\alpha\in(0,2)$. Что происходит в нуле: $\yo x0\arctg\frac{\alpha}x=\frac{\pi}2$. А в бесконечности: $\yo x{+\infty}x\arctg\frac{\alpha}x=\alpha$.
\begin{itemize}
\item Интеграл $\int\limits_0^{\infty}\frac{1}{1+x^2}\,dx=\arctg x\Big|_0^{+\infty}=\frac{\pi}2$ сходится равномерно, так как сходится и подынтегральное выражение не зависит от параметра.
\item Попробую изо всех сил доказать, что всё остальное $f(x,\alpha)=x\arctg\frac{\alpha}x$ "--- монотонная и равномерно ограниченная на $\alpha\in(0,2)$ функция.
\[f'_x(x,\alpha)=\arctg\frac{\alpha}x-x\frac{\alpha}{x^2}\frac1{1+\frac{\alpha^2}{x^2}}=\arctg\frac{\alpha}x-\frac{\alpha x}{x^2+\alpha^2};\]

Получается, что $\forall\ \alpha\in(0,2)\pau \yo x{+\infty}f'(x,\alpha)=0-0=0$.
\begin{multline*}f''_{xx}(x,\alpha)=\frac{-\alpha}{x^2\left(1+\frac{\alpha^2}{x^2}\right)}-\frac{\alpha}{x^2+\alpha^2}+\frac{2\alpha x^2}{(x^2+\alpha^2)^2}=-\frac{2\alpha}{x^2+\alpha^2}+\frac{2\alpha x^2}{(x^2+\alpha^2)^2}=\\=\frac{-2\alpha x^2-2\alpha^3+2\alpha x^2}{(x^2+\alpha^2)^2}=-\frac{2\alpha^3}{(x^2+\alpha^2)^2}<0.\end{multline*}

Отсюда видим, что $\forall\ \alpha\in(0,2)\pau f'_x(x,\alpha)$ монотонно убывает  на множестве $x\in(0,+\infty)$. Значит, $\forall\ \alpha\in(0,2)\pau \exists\ \yo x{+\infty}f'_x(x,\alpha)=\inf\limits_{(0,+\infty)}f'_x(x,\alpha)$. Но мы уже знаем этот предел, значит, мы только что поняли, что $0$ "--- нижняя грань $f'_x(x,\alpha)$ на $(0,+\infty)$. А значит, $f(x,\alpha)$ не убывает.

Когда мы уже знаем, что $f(x,\alpha)$ не убывает, $\forall\ \alpha\in(0,2)\pau \sup\limits_{(0,+\infty)}f(x,\alpha)\hm=\yo x{+\infty} x\arctg\frac{\alpha}x=\alpha<2$. Вот нам и равномерная ограниченность.
\end{itemize}
То есть мы доказали равномерную сходимость интеграла по признаку Абеля.

\item $\ds\int\limits_0^1\frac1{x^{\alpha}}\sin\frac1{x^2}\,dx$ на множестве $\alpha\in(0,3)$. Сделаем замену, не зависящую от параметра $x=t^{-\frac12}$, $dx=\frac{-t^{-\frac32}\,dt}2$. Получаем интеграл $\frac12\int\limits_1^{\infty}\frac1{t^{\frac{\alpha-3}2}}\sin t\,dt$. Докажем от противного, что равномерной сходимости нет. Допустим, сходится равномерно. Тогда выполнено условие критерия Коши, то есть
\[\forall\ \e>0\pau\exists\ b_0\in(1,+\infty)\colon\forall\ b_1,b_2\in(b,+\infty)\pau \bigg|\intop_{b_1}^{b_2}\frac1{t^{\frac{\alpha-3}2}}\sin t\,dx\bigg|<\e.\]
\begin{itemize}
\item Зафиксируем $\e=\frac{\pi}{4\sqrt 2}$.
\item Найдём из критерия Коши $b_0\in(1,+\infty)\colon \forall\ b_1,b_2\in(b,+\infty)\pau \Big|\int\limits_{b_1}^{b_2}\frac1{t^{\frac{\alpha-3}2}}\sin t\,dt\Big|<\e$.
\item Найдём $n\in\N\colon n>\max\left\{2,\frac{b_0}{2\pi}\right\}$.
\item Положим $b_1=2\pi n+\frac{\pi}4$, $b_2=2\pi n+\frac{3\pi}4$, $\alpha=3-2\log_{b_2}2$.
\end{itemize}
Тогда $\forall\ t\in[b_1,b_2]\pau \sin t\in\left[\frac1{\sqrt2},1\right]$. Подынтегральная функция неотрицательна.
\[\intop_{b_1}^{b_2}\frac1{t^{\frac{\alpha-3}2}}\sin t\,dt\ge\intop_{b_1}^{b_2}\frac1{b_2^{\frac{\alpha-3}2}}\frac1{\sqrt2}\,dt=\intop_{b_1}^{b_2}\frac12\cdot\frac1{\sqrt 2}\,dt=\frac{b_2-b_1}{2\sqrt 2}=\frac{\pi}{4\sqrt 2}.\]
%Аргумент синуса бегает по множеству $(0,1)\subset(0,\pi)$. То~есть подынтегральное выражение неотрицательное, можно применить признак сравнения поточечной сходимости. Тогда $\frac1{x^{\alpha}}\sin\frac1{x^2}\sim\frac1{x^{\alpha+2}}$. Значит, интеграл сходится поточечно, если и только если $\alpha+2<1$, то есть только если $\alpha<-1$. На заданном множестве интеграл расходится при любом $\alpha\in(0,3)$.
\end{enumerate}
\item Вычислить
\begin{enumerate}
\item $\ds I=\int\limits_0^{+\infty}e^{-\sqrt[3]{x}}\,dx$. Делаем замену $\sqrt[3]{x}=y$, $x=y^3$, $dx=3y^2\,dy$. $I\hm=3\int\limits_0^{+\infty}e^{-y}y^{3-1}\,dy\hm=3\Gamma(3)\hm=\Gamma(4)\hm=3!\hm=6$.
\item $\ds I(\alpha)=\int\limits_0^1\frac{\ln(1-\alpha^2x^2)}{\sqrt{1-x^2}}\,dx$. Подынтегральное выражение определено для всех $x\in[0,1)$ при $|\alpha|\le1$. Если $I(\alpha)$ определена (если интеграл сходится), то $I(\alpha)=I(-\alpha)$, поэтому будем рассматривать $\alpha\in[0,1]$, причём $I(0)=0$, можно заметить сразу. Но так как позже мы увидим проблемы с производной в нуле, будем рассматривать $I(\alpha)$ на полуинтервале $(0,1]$.
\begin{itemize}
\item $I(1/2)=\int\limits_0^1\frac{\ln\big(1-(1/2)^2x^2\big)}{\sqrt{1-x^2}}\,dx$ "--- интеграл от неположительной функции, можно оценить числитель $\forall\ x\in[0,1)\pau0\ge\ln\left(1-\frac{x^2}4\right)>-\ln\frac43$, а интеграл $\int\limits_0^1\frac{-\ln\frac43}{\sqrt{1-x^2}}\,dx=-\ln\frac43\arcsin x\Big|_0^1=-\frac{\pi}2\ln\frac43$. Значит, в~одной точке интеграл сходится по признаку сравнения.
\item Обозначим подынтегральное выражение через $f(x,a)$. Тогда 
\[f'_{\alpha}(x,\alpha)=\frac{-2\alpha x^2}{(1-\alpha^2x^2)\sqrt{1-x^2}}=\frac2{\alpha}\left(\frac{1-\alpha^2x^2-1}{(1-\alpha^2x^2)\sqrt{1-x^2}}\right)=\frac2{\alpha}\left(\frac1{\sqrt{1-x^2}}-\frac1{(1-\alpha^2x^2)\sqrt{1-x^2}}\right).\]

$\forall\ c,d\in(0,1)\pau f'_{\alpha}(x,\alpha)\in C\big([0,c]\times[d,1]\big)$. Посчитаем интеграл при фиксированном $\alpha\in(0,1)$ с~заменой $x=\sin t$, $dx=\cos t\,dt$.
\begin{multline*}
\intop_0^1f'_{\alpha}(x,\alpha)\,dx=\frac{2\arcsin x}{\alpha}\bigg|_0^1-\frac2{\alpha}\intop_0^{\frac{\pi}2}\frac1{1-\alpha^2\sin^2t}\cdot\frac{\cos t}{\cos t}\,dt=\frac{\pi}{\alpha}-\frac2{\alpha}\intop_0^{\frac{\pi}2}\frac1{\frac1{\sin^2t}-\alpha^2}\cdot \frac1{\sin^2t}\,dt=\\
\cmt{снова замена $u=\ctg t$, $du=\frac{-1}{\sin^2t}\,dt$, $1+u^2=\frac1{\sin^2t}$.}\\
=\frac{\pi}{\alpha}-\frac2{\alpha}\intop_0^{+\infty}\frac{du}{1+u^2-\alpha^2}=\frac{\pi}{\alpha}-\frac{2\arctg\frac{u}{\sqrt{1-\alpha^2}}}{\alpha\sqrt{1-\alpha^2}}\bigg|_0^{+\infty}=\frac{\pi}{\alpha}\left(1-\frac1{\sqrt{1-\alpha^2}}\right).
\end{multline*}
Установим равномерную сходимость интеграла от производной на произвольном вложенном отрезке вида $\alpha\in[d,1-d]\subset(0,1)$, где $d\in(0,1/2)$.
\[\left|\frac2{\alpha}\left(\frac1{\sqrt{1-x^2}}-\frac1{(1-\alpha^2x^2)\sqrt{1-x^2}}\right)\right|\le\frac2{d}\bigg|\frac2{\sqrt{1-x^2}}\bigg|\]
Построили мажорирующую функцию, установили равномерную сходимость по признаку Вейерштрасса. Правда на вложенных отрезка.
\end{itemize}

Получается, что на любом отрезке вида $\alpha\in[d,1-d]\subset (0,1)$ функция $I(\alpha)$ существует, дифференцируема и $I'(\alpha)=\int\limits_0^1f'_{\alpha}(x,\alpha)\,dx=\frac{\pi}{\alpha}\left(1-\frac1{\sqrt{1-\alpha^2}}\right)$. Ну значит, всё на том же произвольном отрезке 
\begin{multline*}I(\alpha)\hm=\int\frac{\pi}{\alpha}\left(1-\frac1{\sqrt{1-\alpha^2}}\right)\,dx=\pi\ln\alpha-\pi\int\frac1{\alpha^2\sqrt{\frac1{\alpha}-1}}\,d\alpha=\\
=\pi\left(\ln\alpha+\ln\left(\frac1{\alpha}+\sqrt{\frac1{\alpha^2}-1}\right)\right)+C=\pi\left(\ln\big(1+\sqrt{1-\alpha^2}\big)\right)+C.\end{multline*}
Теперь надо найти константу. Когда оценивался $I(1/2)$, делал примерно то же самое: \[\forall\ |\alpha|<\frac12,\ \forall\ x\in[0,1)\pau 0\ge\ln\left(1-\alpha^2x^2\right)>-\ln\frac43;\imp \left|\frac{\ln\left(1-\alpha^2x^2\right)}{\sqrt{1-x^2}}\right|\le\frac{\ln4-\ln3}{\sqrt{1-x^2}}.\]
Делаем третий вывод из сходимости интеграла $\int\limits_0^1\frac{\sqrt{1-x^2}}=\arcsin x\Big|_0^1=\pi/2$. На этот раз можем сказать, что по признаку Вейерштрасса $I(\alpha)$ сходится равномерно на $\alpha\in(-1/2,1/2)$. Значит, $I(\alpha)$ непрерывна, например, в точке $\alpha=0$. То есть \[I(0)=\yo{\alpha}0I(\alpha)=\yo{\alpha}{0+}I(\alpha)=\yo{\alpha}{0+}\pi\left(\ln\big(1+\sqrt{1-\alpha^2}\big)\right)+C; \imp C=-\pi\ln2.\]
Осталось учесть, что функция чётная. Значит, $\forall\ \alpha\in(-1,1)\pau I(\alpha)=\pi\left(\ln\big(1+\sqrt{1-\alpha^2}\big)\right)-\pi\ln2$.
\end{enumerate}
\end{enumerate}

\end{document}