[Формула $\overset{\text{Тейлора}}{\cancel{\text{члена}}}$ с~остаточным членом в~общем виде (в общей форме)]
    Пусть функция $f$ $n$ раз непрерывно дифференцируема на~отрезке с~концами $x_0,x$ и $n+1$ раз просто как-нибудь дифференцируема на~интервале с~концами
    $x_0,x$
    (функция должна быть всюду хорошая). Пусть также функция $\phi$ непрерывна на~отрезке и дифференцируема на~интервале с~концами $x_0,x$, причём
    $\phi'$ не обращается в~ноль на~этом интервале.
    Тогда $\exists\  \xi$, лежащая между $x$ и $x_0$ такая, что
    $f(x)\hm=\underbrace{\pn fx0n{x_0}}_{P_n(x,f)}+\underbrace{\vphantom{\pn fx0n{x_0}}\dfrac{f^{(n+1)}(\xi)(x-\xi)^n}{n!\phi'(\xi)}\cdot\big(\phi(x)-\phi(x_0)\big)}_{r_n(x)}$
