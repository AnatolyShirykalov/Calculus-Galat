
Если \(\ol x_0\) "--- точка условного экстремума функции \(f\), повторю, не жалко, при условиях \({\Phi_1=0,\ldots,\Phi_N=0}\), снова не жалко: функции \(f,\Phi_1,\ldots,\Phi_N\) непрерывно
дифференцируемы в окрестности точки \(\ol x_0\) и
\[\rk \bigg(\CP{\Phi_j}{x_k}\bigg)\bigg|_{\ol x_0} = \rk 
\left.\begin{pmatrix}
\CP{\Phi_1}{x_1} & \dots & \CP{\Phi_1}{x_{M+N}} \\
\vdots & \ddots & \vdots \\
\CP{\Phi_N}{x_1} & \dots & \CP{\Phi_N}{x_{M+N}}
\end{pmatrix}\right|_{\ol x_0} = N,\]
то \(\exists\ \lambda_1^0,\ldots,\lambda_N^0\in\R\colon x_1^0,\ldots,x_{M+N}^0,\lambda_1^0,\ldots,\lambda_N^0\) являются решением системы
\[\begin{cases}
\CP L{x_1} = 0, \ldots, \CP L{x_{M+N}}=0, \\
\CP L{\lambda_1}=\Phi_1=0, \ldots, \CP L{\lambda_N}=\Phi_N=0.
\end{cases}\]
