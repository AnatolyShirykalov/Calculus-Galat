
	 Пусть $A$ "--- л-связное подмножество $\R^M$, скалярная функция $f\in C(A)$. Тогда $f$ принимает все промежуточные значения на $A$.
	 То есть $\forall\ \ol x_1,\ol x_2\in A$, $\alpha$ "--- числа, лежащего между $f(\ol x_1)$ и $f(\ol x_2)$,
	 найдётся $\ol x_3\in A\colon f(\ol x_3) = \alpha$.
	 