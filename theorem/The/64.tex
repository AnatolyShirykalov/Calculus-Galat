[Первая теорема о среднем\footnote{В дифференциальном исчислении уже были теоремы (Ролля: \ref{Roll}, Лагранжа \ref{Lag}, Коши \ref{Kohi}) о среднем.}]
 	\label{FirTheMid}Может быть, первая и единственная, может и не единственная. Формулировать дольше, чем доказывать. Пусть $f,g\in\Rim[a,b], g\geq0$ на $[a,b]$. Тогда
 	$\exists\  C\colon\inf\limits_{[a,b]}f(x)\leq C\leq \sup\limits_{[a,b]}f(x)$ и
 	$$\ds\int\limits_a^bf(x)g(x)\,dx=C\cdot\int\limits_a^bg(x)\,dx.$$
 	Интеграл произведения равен интегралу неотрицательной функции умножить на, вот проблема, как обычно неизвестную константу.
 	