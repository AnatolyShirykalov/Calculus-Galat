[О неявном отображении] \label{ThoNO} Пусть
\begin{itemize}
\item При всех $n\in\{1,\ldots,N\}\pau F_n(\ol x,\ol y)$ непрерывно дифференцируема в окрестности точки $(\ol x_0,\ol y_0)$;
\item Сама точка $(\ol x_0,\ol y_0)$ является
решением системы $F_n(\ol x_0,\ol y_0)=0$;
\item Рассмотрим матрицу:
\[\begin{pmatrix}
\CP{F_1}{y_1} & \CP{F_1}{y_2} & \dots & \CP{F_1}{y_N} \\[1ex]
\CP{F_2}{y_1} & \CP{F_2}{y_2} & \dots & \CP{F_2}{y_N} \\
\vdots        & \vdots        & \ddots&  \vdots \\
\CP{F_N}{y_1} & \CP{F_N}{y_2} & \dots & \CP{F_N}{y_N} \\
\end{pmatrix}\]
"--- получается, как говорят, функциональная матрица. Что хорошего в этой матрице? Она квадратная. Значит, можно посчитать определитель, иногда говорят «функциональный определитель».
Матрицу называют матрицей Якоби, её определитель "--- Якобиан.
Пусть Якобиан в точке $(\ol x_0,\ol y_0)$ отличен от~нуля. Это чистый аналог условия $F'(\ol x_0,y_0)\ne 0$ в предыдущей теореме.
\end{itemize}
Тогда $\exists\ \e_0>0,\ \exists\ \delta_0>0\colon \forall\ \ol x\in B_{\delta_0}(\ol x_0)\pau\exists!\ \ol y = \ve yN\in B_{\e_0}(\ol y_0)\colon$
\[\begin{cases}
F_1(\ol x,\ol y) = 0;\\
F_2(\ol x,\ol y) = 0;\\
\dots\\
F_N(\ol x,\ol y) = 0.
\end{cases}\]
Полученные таким образом функции $y_1(\ol x),\ldots,y_N(\ol x)$ непрерывно дифференцируемы в~окрестности точки~$\ol x_0$.
