
	 Пусть $f\ve xM$ дифференцируема в точке $\ol x_0 \hm= \ve{x^0}M$, а ещё есть $\phi_1\ve tK$, дифференцируемая в точке $\ol t_0 \hm= \ve{t^0}K$ и такая, что
	 $\phi_1(\ol t_0) \hm= x_1^0$, есть $\phi_2\ve tK$, дифференцируемая в точке $\ol t_0$ и такая, что $\phi_2(\ol t_0) \hm= x_2^0,\ldots,$ 
	 есть $\phi_M\ve tK$, дифференцируемая в точке $\ol t_0$ и такая, что $\phi_M(\ol t_0) \hm= x_M^0$.
	 Всё это позволяет рассмотреть сложную функцию.
	 Тогда $f\big(\phi_1(\ol t),\phi_2(\ol t),\ldots,\phi_M(\ol t)\big)$ дифференцируема в точке $\ol t_0$, причём 
	 (зная частные производные восстановим дифференциал)
	 \[\CP{f\big(\phi_1(\ol t),\ldots,\phi_M(\ol t)\big)}{t_k}\bigg|_{\ol t_0} \hm= \CP{f(\ol x)}{x_1}\bigg|_{\ol x_0}\CP{\phi_1(\ol t)}{t_k}\bigg|_{\ol t_0} +
	 \CP{f(\ol x)}{x_2}\bigg|_{\ol x_0}\CP{\phi_2(\ol t)}{t_k}\bigg|_{\ol t_0}+\ldots+\CP{f(\ol x)}{x_M}\bigg|_{\ol x_0}\CP{\phi_M(\ol t)}{t_M}\bigg|_{\ol t_0}.\]
	 