 Простенькая теорема:

  \begin{enumerate}
    \item Если $F(x)$ "--- точная первообразная функции $f(x)$ на~$I$ и $C\in\R$, то $F(x)+C$ тоже является точной первообразной функции $f(x)$ на~$I$ (доказывать нечего).

    \item Если $F_1(x)$ и $F_2(x)$ "--- точные первообразные функции $f(x)$ на~$I$, то $\exists\  C\in\R\colon F_2(x)\hm=F_1(x) + C$ всюду на~промежутке $I$.
  \end{enumerate}

