[Теорема о неявной функции]\label{ThoNF}
	Пусть для некоторого $\gamma_0>0\pau F(\ol x,y)$ непрерывно дифференцируема в $\gamma_0$-окрестности точки $(\ol x_0,y_0)$, причём
	$F(\ol x_0,y_0) = 0$ и $F'_y(\ol x_0,y_0)\ne 0$ "--- наоборот\footnote{Следовательно,
	$F'_y$ будет строго больше нуля или строго меньше нуля в целой окрестности точки $(\ol x_0,y_0)$ в силу того, что все частные
	производные непрерывны. Следовательно, $F(\ol x,y)$ в этой окрестности при фиксированном $\ol x$ как функция от $y$ является строго монотонной.}.
	Тогда по предыдущей теореме в окрестности точки $(\ol x_0,y_0)$ уравнение $F(\ol x,y) = 0$ однозначно определяет функцию $y(\ol x)$ (пока ничего нового не сказал,
	просто применил предыдущую теорему). Новая часть утверждения: эта функция опять же локально в этой окрестности непрерывно дифференцируема и 
	\[\CP{y}{x_j}\bigg|_{\ol x} = -\frac{F'_{x_j}\big(\ol x,y(\ol x)\big)}{F'_y\big(\ol x,y(\ol x)\big)}.\]
	