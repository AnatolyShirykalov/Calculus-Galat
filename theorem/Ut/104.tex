\label{I3}
	Пусть $x(t)$ "--- дифференцируемое отображение промежутка $J$ на~промежуток $I$, производная которого на~всём промежутке $J$ не обращается в~ноль
	(значит, по~правилу \ref{Darbu} Дарбу либо всюду $\hm>0$, либо всюду $\hm<0\ \hm{\imp}$ строго монотонная функция: в~этих условиях у нас существует обратное отображение $t(x)$).
	Пусть также на~$J$ интеграл $\ds\int f\big(x(t)\big)x'(t)\,dt\hm=F(t)$.


$\a$Буква $t$ перегружена, но вводить какие-то новые обозначения $t\hm=\phi(x)$ нет смысла. Величины всё равно одни и те же.$\s$

Тогда на~$I$ интеграл $\ds\int f(x)\, dx\hm=F\big(t(x)\big)$. В~некотором смысле это замена переменных в~другую сторону.

