
    $\forall\  x\in\R\pau \sin x\hm=x-\dfrac{x^3}{3!}+\dfrac{x^5}{5!}-\dfrac{x^7}{7!}+\ldots\hm=\sum\limits_{k\hm=0}^\infty(-1)^k\dfrac{x^{2k+1}}{(2k+1)!}$.

    $\cos x\hm=\sum\limits_{k\hm=0}^\infty\dfrac{(-1)^kx^{2k}}{(2k)!}$

    а~также: какие производные от $e^x$? Ровно такие же $e^x$, тогда если $x\in(-C,C)$, то $f^{(n)}(x)\leq e^C$. Значит,
     $\forall\  x\in(-C,C)\pau e^x\hm=1+\dfrac{x}{1!}+\dfrac{x^2}{2!}+\ldots\hm=\sum\limits_{k\hm=0}^\infty\dfrac{x^k}{k!}$, но $C$ можно брать любую, значит это
    утверждение верно для $x\in\R$, а~грань $C$ нужна только для доказательства.
