
	 Пространство со скалярным произведением "--- это упорядоченная пара (как обычно: множество и действие) $\big(L,(\cdot,\cdot)\big)$, где
	 $L$ "--- линейное пространство над $\R$, а вот эта штука: $(\cdot,\cdot)$ "--- скалярное произведение, то есть 
	 отображение, сопоставляющее паре $\ol x,\ol y\in L$ действительное число, удовлетворяющее следующим свойствам:
	 \begin{enumerate}
	   \item $(\ol x,\ol y) = (\ol y,\ol x)$ "--- это симметричность;
	   \item $(\alpha,\ol x,\ol y) = \alpha (\ol x,\ol y)$ "--- первая половина линейности по первому аргументу (однородность);
	   \item $(\ol x_1 + \ol x_2,\ol y) = (\ol x_1,\ol y)+(\ol x_2,y)$ "--- вторая половина линейности (аддитивность). Симметричность влечёт линейность по второму аргументу;
	   \item $\forall\ x\in L\pau (\ol x,\ol x)\geq0$, причём $(\ol x,\ol x)=0\iff\ol x=0$.
	 \end{enumerate}
	 