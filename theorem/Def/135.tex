
 	Система $\B$ подмножеств $X$ называется базой, если
 	\begin{enumerate}
 	  \item Пустое множество не принадлежит этой системе множеств ($\q\nin \B$);
 	  \item И во-вторых, для любых двух его элементов базы найдётся третий элемент, который лежит в~пересечении: $\forall\  B_1,B_2\in \B
 	  \pau \exists\  B_3\in\B\colon B_3\subset B_1\cap B_2$.
 	\end{enumerate}
 