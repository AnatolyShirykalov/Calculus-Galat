
	Нормированное пространство "--- это упорядоченная пара $\big(L,\|\cdot\|\big)$, где 
	$L$ "--- линейное пространство (над $\R$), а $\|\cdot\|$ "--- норма, то есть отображение элементов $L$ в $\R^+$, 
	удовлетворяющее следующим свойствам:
	\begin{enumerate}
	  \item $\forall\  \alpha\in\R,\ \forall\ \ol x\in L\pau\|\alpha\ol x\|=|\alpha|\|\ol x\|$\footnote{Из этого следует, что $\|\ol0\|=0$.};
	  \item \label{leTr} $\forall\ \ol x,\ol y\in L\pau \|\ol x+\ol y\|\leq\|\ol x\|+\|\ol y\|$;
	  \item $\|\ol x\|=0\iff\ol x=\ol 0$ "--- это разные нули.
	\end{enumerate}
	В нормированном пространстве норма порождает метрику: $\rho(\ol x,\ol y)=\|\ol x - \ol y\|$. Однако есть метрика, которая нормой никак не задаётся (например, опять метрика Гарри Поттера).
	