
   	Пусть $f\colon[a,b)\to\R$ ($a\in\R,\ b\in\ol\R$). Если $\forall\  \tilde b\in[a,b)\pau f\in\Rim[a,\tilde b]$ и существует $\lim\limits_{\tilde b\to b-0}
   	\ds\int\limits_a^{\tilde b}f(x)\,dx=I\in\ol\R$, то
   	$I$ называют несобственным интегралом функции $f$ по полуотрезку $[a,b)$ (с~особенностью в точке $b$). Аналогично с особенностью в точке $a$.
   	
   	Если $I\in\R$, то говорят, что функция $f$ интегрируема в несобственном смысле на $[a,b)$ или $\ds\int\limits_a^bf(x)\,dx$ сходится.
   