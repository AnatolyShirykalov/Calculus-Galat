
    Функция $f$ дифференцируема в~точке $x_0$, если $\Delta f\hm=f(x)-f(x_0)$ представляется в~виде $\Delta f\hm=A\cdot\Delta x+\lims{\oo(\Delta x)}{\Delta x\to 0}{(x\to x_0)}$, где $A\in \R$.

    Эквивалентно: $\Delta f-A\cdot\Delta x\hm=\underset{\Delta x\to0}{\oo(\Delta x)}$ и $\Delta f\hm=\Delta x\big(A+\underset{\Delta x\to 0}{\oo(1)\big)}$ "--- приращение функции есть приращение аргумента почти умноженное на~константу.


При этом линейный оператор $\R\to\R\colon \Delta x\to A\Delta x$ (очень умный оператор) называется дифференциалом функции $f$ в~точке $x_0$.
Обозначается $df\big|_{x_0}$ (при больших $\Delta x$ выдаёт фигню какую-то, главное, что линейную).

 