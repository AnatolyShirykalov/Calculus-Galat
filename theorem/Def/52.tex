
    $\e$-окрестностью точки $x$ называется интервальчик $(x-\e,x+\e)$. Обозначается $B_\e(x)$.

    Проколотая $\e$-окрестность точки $x\colon(x-\e,x+\e)\dd\{x\}$. Обозначается $B'_\e(x)$.

    На~$\R\colon B_\e(+\infty)\hm=B'_\e(+\infty)\hm=\big(\frac{1}{\e},+\infty\big)$ "---
    \textup{(}$\frac 1\e$ чтобы при увеличении $\e$ окрестность сужалась\textup{)}. $B_\e(-\infty)\hm=(-\infty,-\frac 1\e)$.

    На~$\ol\R\colon B_\e(+\infty)\hm=\big(\frac{1}{\e},+\infty\big],B'_\e(+\infty)\hm=\big(\frac{1}{\e},+\infty\big)$. Для $-\infty$ аналогично.
