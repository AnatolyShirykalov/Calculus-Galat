
Функция называется непрерывной в~точке $a$ по~множеству $A$, если

\begin{enumerate}
    \item $\exists\  \delta_0\hm>0\colon f(x)$ определена в~$B_{\delta_0}(a)\cap A$ и
        в~самой точке $a$;

    \item $\forall\ \e\hm>0\pau\exists\ \delta\hm>0\colon \forall\  x\in B'_{\delta}(a)\cap A \pau |f(x)-f(a)|\hm<\e$, уже выясняли, что неважно: проколотая окрестность, не проколотая$\ldots$
\end{enumerate}

Почему необходимы оговорки? $a$ обязана являться предельной точкой $A$.

$\lim\limits_{A\ni x\to a}f(x)\hm=f(a)$.
