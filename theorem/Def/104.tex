
 Пусть функция $f$ определена на~$A$. Модулем непрерывности на~множестве $A$ называется функция 
 $\omega_f(\delta)$, которая отображает: $(0;+\infty)\to[0,+\infty)$. Собственно $\omega_f(\delta)\hm=
 \sup\limits_{\begin{smallmatrix}x,\tilde x\in A\\|x-\tilde x|\hm<\delta\end{smallmatrix}}|f(x)-f(\tilde x)|$ 
 "--- насколько максимально может меняться значение функции, если аргумент меняется на~$\delta$.
 Иногда под~$\sup$ пишут $|x-\tilde x|\leq\delta$, для непрерывности разницы никакой, дело вкуса. 
 Считаем также, что $A$ непусто, иначе $\sup \q\hm=-\infty$.
 