\label{vrost}
        $f(x)$ возрастает (иногда говорят, строго возрастает) на~$A$, если $\forall\  x_1,x_2\in A\ x_1\hm<x_2\hm{\imp} f(x_1)\hm<f(x_2)$.

        Обозначим $\nearrow$;

        Неубывает $f(x_1)\leq f(x_2) \ \nv$;

        Убывает $f(x_1)\hm>f(x_2)$ $\searrow$;

        Невозрастает (убывает в~нестрогом смысле) $f(x_1)\geq f(x_2)$ $\nuu$.

        Если функция $\searrow$ или $\nearrow$, то она называется строго монотонной на~$A$. Если функция $\searrow,\nearrow,\nuu$ или $\nv$, то она называется нестрого монотонной.
    