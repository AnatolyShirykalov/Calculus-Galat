
    Пусть $A\subset \R$ и $\underset{a\in \ol{\R}}{a\text{ "--- преде}}$льная точка множества $A$. % Это шедевр. Я~долго смеялся. —А. В.—

    Предел $f(x)$ при $x\to a$ по~множеству $A$ равен $l\ (l\in\ol{\R})$.

    Начинаю делать для вас страшное: $\lim\limits_{A\ni x\to a} f(x)\hm=l$, если

    \begin{enumerate}
        \item $\exists\ \delta_0\hm>0\colon f(x)$ определена в~$B'_{\delta_0}(a)\cap A$;

        \item \begin{enumerate} \item $\forall\ \e\hm>0\ \exists\ \delta\hm>0\colon\forall\  x\in B'_\delta(a)\cap A\pau f(x)\in B_\e(l)$;

        \item $\forall$ последовательности точек $\big\{x_n\big\}_{n\hm=1}^\infty$, которая удовлетворяет условиям:

        \begin{enumerate}
            \item $f$ определена во всех $x_n$;

            \item $x_n\xrightarrow[n\to \infty]{}a\pau x_n\in A\dd\{a\},$

            $$f(x_n)\xrightarrow[n\to \infty]{}l.$$
        \end{enumerate}

        \end{enumerate}


    \end{enumerate}

    