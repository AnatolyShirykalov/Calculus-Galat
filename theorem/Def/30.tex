
    Натуральные числа $\big(\mathbb{N}\big)$ "--- это упорядоченная пара $(N,S)$, где $N$ "--- некоторое множество, $S$ "--- отношение
     на~множестве $N$, называемое \textbf{отношением следования}, удовлетворяющее следующим свойствам (собственно аксиомам Пеано):

    \begin{enumerate}
        \item Отношение $S$ является отображением, то есть за каждым элементов $N$ следует ровно один элемент;

        \item Отображение $S$ инъективно, то есть каждое натуральное число следует не более чем за одним;

        \item Существует элемент во множестве $N$, не следующий ни каким другим (обозначать мы его будем
            $1$\footnote{Натуральные числа можно, в~общем-то, начинать и с~двойки. 
            В~Америке вот принято $0$ тоже считать натуральным числом. 
            Но вот мы можем делить на~все натуральные числа, а~они нет. 
            Видите, какие мы продвинутые, а~они убогие!}), то есть $S$ не сюръективно;



        \item\label{Induct} Аксиома индукции: пусть $M\subset N$ удовлетворяет следующим свойствам:
            \begin{enumerate}
                \item $1\in M$;
                \item Если элемент множества $N$ принадлежит $M$, то следующий за ним также принадлежит $M$.
            \end{enumerate}
            Тогда $M\hm=N$.
    \end{enumerate}
