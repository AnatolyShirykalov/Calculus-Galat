
Точка \(\ol x_0\in\R^{M+N}\) называется точкой строгого условного минимума функции \(f\) при~условиях \(\Phi_1=0,\dots,\Phi_N=0\), если
\begin{itemize}
\item \(\ol x_0\in R\), \(f(\ol x)\) определена в точке \(\ol x_0\) и
\item \(\exists\ \delta>0\colon \forall\ \ol x\in B'_\delta(\ol x_0)\cap R\pau f(\ol x)\) определена и \(f(\ol x) > f(\ol x_0)\).
\end{itemize}
