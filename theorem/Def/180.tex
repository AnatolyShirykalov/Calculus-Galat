\label{DIFF}
	 $f(\ol x)$ дифференцируема в точке $\ol x_0$, если $f$ определена в некоторой окрестности точки $\ol x_0$ и
	 $f(\ol x) - f(\ol x_0)$ близко к~линейному. Давайте это формализовывать:
	 если ещё $\exists$ линейное отображение $L\colon \R^M\to \R\colon f(\ol x) - f(\ol x_0) = L(\ol x - \ol x_0) + \alpha (\ol x - \ol x_0)$,
	 где $\alpha(\ol x- \ol x_0) = \ooob{\|\ol x - \ol x_0\|}{\ol x\to \ol x_0}$, то есть $\ds\frac{\alpha(\ol x- \ol x_0)}{\|\ol x - \ol x_0\|}\tend{\ol x\to \ol x_0}0$.
	 
	 Естественно $L(\ol x - \ol x_0)$ называется дифференциалом $f(\ol x)$ в точке $\ol x_0$. 
	 