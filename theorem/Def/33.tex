
Действительными числами называется упорядоченная четвёрка $(\Ri,+,\bullet,\leq)$, где $\Ri$ "--- множество,
 $+$ и $\bullet\colon\ \Ri^2\to\Ri$, $\leq$ "--- отношение между элементами $\Ri$, причём выполнены все аксиомы 1--14 (\ref{1-14}), а~также
\begin{enumerate}\setcounter{enumi}{14}\renewcommand{\theenumi}{$\arabic{enumi}$}
\item \label{dedeking} Аксиома полноты Дедекинда: Пусть $A$ и $B$ "--- непустые подмножества $\Ri$, причём для любых элементов $a\in A$, $b\in B$ справедливо $a\leq b$. Тогда $\exists\  c\in \Ri\colon\forall\  a\in A,\forall\  b\in B\pau a\leq c\leq b$.
\end{enumerate}

   \pic{galat/5/dedeking}{2}{depic}{По сути заткнули дырки}
 Обозначение $\R$: не возникает путаницы между множеством $\Ri$ и структурированным множеством $\R$.
 \[
    \R\supset\Q\supset\Z\supset\N
 \]

Действительные числа определяются требованием свойств!
