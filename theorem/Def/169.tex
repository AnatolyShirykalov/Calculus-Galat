
	 Октрытым шаром $B_r(\ol x)$ с~центром в~точке~$\ol x\in\R^M$ радиуса~$r$ называется множество всех точек $\ol y\in\R^M$,
	 удовлетворяющих условию: $\rho(\ol x,\ol y) < r$\footnote{Естественно это определение можно давать для любого метрического пространства.}.
	 \[ B_r(\ol x) = \big\{\ol y\in\R^M\colon \rho(\ol x,\ol y)<r\big\}.\]
	 Аналогично: замкнутым шаром назовём $\ol B_r(\ol x) = \big\{\ol y\in\R^M\colon\rho(\ol x,\ol y)\leq r\big\}$\footnote{То, что шар называется замкнутым или открытым, ещё не значит, что 
	 когда мы дадим определение открытого множества и определение замкнутого множества, открытый шар окажется открытым, а замкнутый "--- замкнутым. Нужно будет убедиться.}.
	 