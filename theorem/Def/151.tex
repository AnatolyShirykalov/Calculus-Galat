
 	Колебинием (осцилляцией) функции $g$ на отрезке $\Delta$ называется следующая величина $\sup\limits_{t_1,t_2}\big|g(t_1)-g(t_2)\big|$.
 	Обозначается по-разному. Иногда так и пишут $\osc(g,\Delta)$. Иногда $w(g,\Delta)$, иногда $\omega(g,\Delta)$. Омегу можно путать с модулем непрерывности.
 