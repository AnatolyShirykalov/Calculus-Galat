
 	Функция $f$ кусочно непрерывна на отрезке $[a,b]$, если она непрерывна на $[a,b]$ всюду, за~исключением
 	конечного набора точек, в которых у функции $f$ имеются устранимые разрывы или разрывы первого рода (рисунок~\ref{urpr}).
 	
 	\pic{galat/41/01}{2}{urpr}{Кусочно непрерывная функция}
 	
 	Иными словами: $f$ кусочно непрерывна на отрезке $[a,b]$, если $\exists$ разбиение отрезка $[a,b]$ на конечный набор
 	непересекающихся подотрезков $\Delta_1,\ldots,\Delta_N\colon\forall\  j\in\{1..N\}\pau f$ можно
 	доопределить или переопределить в концах отрезка $\Delta_j$ так, что $f$ станет непрерывна на~$\Delta_j$.
 