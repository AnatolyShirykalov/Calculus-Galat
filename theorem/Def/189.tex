
	 Пусть частная производня $\ds\CP f{x_j}$ определена в~некоторой окрестности точки $\ol x_0$, неважно, дельта, не дельта: $B_\delta(\ol x_0)$.
	 Тогда в~этой окрестности $\ds\CP f{x_j}$ можно рассматривать, как самую обычную скалярную функцию нескольких переменных. Если у~этой функции существует 
	 производная по~переменной~$x_k$, то~она~называется второй производной по~переменным $x_j,x_k$ функции $f$ в точке $\ol x_0$.
	 
	 Нужно определиться с обозначениями: $\ds\underbrace{\overbrace{\CP{ }{x_k}}^{\substack{\text{потом}\\ \text{навешивается} \\ \text{пр-ая по }x_k}}
	 \mspace{-3mu}\text{от}\mspace{-3mu}\overbrace{\CP f{x_j}}^{\substack{\text{сначала} \\ \text{считается} \\ \text{пр-ая по }x_j}}}_{\substack{\text{традиционный порядок} \\ \text{чтения слов: справа налево}}}$
	  обычно обозначают через $\ds\CP{^2f}{x_k\partial x_j}$ или $\big(f'_{x_j}\big)'_{x_k} \hm= f''_{x_jx_k}$.
	  Если $j=k$, то производная называется чистой, а когда переменные разные "--- смешанной. Обозначения для чистых: $\ds\frac{\partial^2f}{(\partial x_j)^2}\hm= 
	  \CP{^2f}{x_j^2}\hm=f''_{x_j^2}$.  Если не оговорено противное, квадрат относится ко всему $\partial x_j$.
	 